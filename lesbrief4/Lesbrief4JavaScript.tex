\documentclass[a4paper]{report}

\usepackage{NielsPackage}

\lstset{language=HTML}

\hypersetup{
	pdfauthor = {Niels Doorn},
	pdftitle = {Javascript Lesbrief 4},
	pdfsubject = {HTML, CSS, JavaScript},
	pdfkeywords = {HTML,css,lesbrief},
	pdfcreator = {NielsDoorn/RocVanTwente}
}

\rhead{\textsc{Javascript Lesbrief 4}}
\lhead{}
\chead{}
\lfoot{Niels Doorn \copyright~2013}
\cfoot{}
\rfoot{\thepage}

\fancypagestyle{plain}{
	\fancyhf{}
	\fancyfoot[L]{Niels Doorn \copyright~2013}
	\fancyfoot[C]{}
	\fancyfoot[R]{\thepage}
	\renewcommand{\headrulewidth}{0pt}
	\renewcommand{\footrulewidth}{0.4pt}
}

\begin{document}

\chapter*{\textcolor{seccol}{Javascript} Lesbrief 4}

\section*{Beoordeling}
Deze lesbrief bestaat uit een aantal vragen en opdrachten. Deze vragen moet je in een document beantwoorden en samen met de uitwerkingen van de opdrachten af laten tekenen door de docent. Voor de eindopdracht krijg je een cijfer.

\section*{Geolocating}
Een van de nieuwe HTML5 features is geolocating. Hiermee kun je bepalen op welke plaats je website bezoeker zich bevindt. Dit kan heel handig zijn als je bijvoorbeeld rekening wilt houden met de context waarin iemand een website bezoekt. Bijvoorbeeld: een pizza keten heeft een website die je direct aan kan geven waar de dichtsbijzijnde pizzeria is. Of een website met file-informatie die direct aangeeft in welke file je staat.

\subsection*{Opdracht 1: theorie}
Lees hoofdstuk 5 van het boek door en beantwoord de volgende vragen. 

\begin{enumerate}
	\item Noem vier manieren waarop geolocating achter de positie van een bezoeker kan komen.
	\item Bedenk drie voorbeelden van websites waarbij het gebruik van geolocating toegevoegde waarde heeft.
	\item Wat zijn de nederlandse woorden voor ``latitude'' en ``longitude''?
	\item Wat zijn de coordinaten van onze school?
	\item Wat zijn radialen? Wat zijn graden? Hoe kun je graden omrekenen naar radialen? En omgekeerd?
\end{enumerate}

\subsection*{Opdracht 2: Waar ben ik?}
Maak een webpagina die je huidige coordinaten laat zien. Je kunt hiervoor de instructies in het boek volgen. Test het ook op een mobiele telefoon.

\subsection*{Opdracht 3: Op de kaart}
Pas de vorige webpagina aan zodat je op een kaart ziet waar je bent\footnote{Dit mag ook met Open Street Maps in plaats van Google maps}.

\begin{itemize}
	\item Geef met een marker aan op welke plaats je bent.
	\item Zorg ervoor dat de webpagina er mooi uitziet, dus niet alleen een kaart, maar maak er een totaalproduct van.
\end{itemize}

\subsection*{Eindopdracht: Hoe ver ben je van een bepaald punt}
Pas de vorige webpagina aan met de volgende uitbreiding:
\begin{itemize}
	\item Voeg nog een marker toe op een speciale plaats, bijvoorbeeld je eigen huis of de Oude Markt in Enschede
	\item Bepaal met JavaScript hoe ver de website bezoeker verwijdert is van die speciale plaats.
	\item Zorg er voor dat de website verandert als je dichterbij bent dan bijvoorbeeld 1km.
\end{itemize}

\subsection*{Bonusopdracht: Volg de bezoeker}
Pas de vorige webpagina zo aan dat je continue de bezoeker kunt volgen\footnote{Dit staat beschreven in het boek}.

\end{document}